\documentclass[a4paper]{article}
% generated by Docutils <http://docutils.sourceforge.net/>
\usepackage{fixltx2e} % LaTeX patches, \textsubscript
\usepackage{cmap} % fix search and cut-and-paste in Acrobat
\usepackage{ifthen}
\usepackage[T1]{fontenc}
\usepackage[utf8]{inputenc}
\usepackage{amsmath}
\usepackage{color}
\setcounter{secnumdepth}{0}

%%% Custom LaTeX preamble
\usepackage{units, amsmath, amsfonts, amssymb, textcomp, gensymb, marvosym,wasysym}
\usepackage[left=2cm,right=2cm,top=1cm,bottom=1.5cm]{geometry}
\usepackage[version=3]{mhchem}
\usepackage{tocloft}
\usepackage{array}
\usepackage{fancyhdr}
\usepackage{fancybox,shadow}
\usepackage{setspace}
\fancypagestyle{myplain}
{
\fancyhf{}
\renewcommand\headrulewidth{0pt}
\renewcommand\footrulewidth{0pt}
\fancyfoot[C]{\thepage}
}
\pagestyle{myplain}
%\pagestyle{empty}
\setlength{\parskip}{\baselineskip} % Extra line between paragraphs
\setlength{\parindent}{0pt} % No indent at the start of paragraphs
\fontsize{12pt}{14.4pt}
\selectfont
\hyphenation{Win-kel-ge-schwin-dig-keit}

%%% User specified packages and stylesheets

%%% Fallback definitions for Docutils-specific commands

% admonition (specially marked topic)
\providecommand{\DUadmonition}[2][class-arg]{%
  % try \DUadmonition#1{#2}:
  \ifcsname DUadmonition#1\endcsname%
    \csname DUadmonition#1\endcsname{#2}%
  \else
    \begin{center}
      \fbox{\parbox{0.9\textwidth}{#2}}
    \end{center}
  \fi
}

% rubric (informal heading)
\providecommand*{\DUrubric}[2][class-arg]{%
  \subsubsection*{\centering\textit{\textmd{#2}}}}

% title for topics, admonitions, unsupported section levels, and sidebar
\providecommand*{\DUtitle}[2][class-arg]{%
  % call \DUtitle#1{#2} if it exists:
  \ifcsname DUtitle#1\endcsname%
    \csname DUtitle#1\endcsname{#2}%
  \else
    \smallskip\noindent\textbf{#2}\smallskip%
  \fi
}

% transition (break, fancybreak, anonymous section)
\providecommand*{\DUtransition}[1][class-arg]{%
  \hspace*{\fill}\hrulefill\hspace*{\fill}
  \vskip 0.5\baselineskip
}

% hyperlinks:
\ifthenelse{\isundefined{\hypersetup}}{
  \usepackage[colorlinks=true,linkcolor=blue,urlcolor=blue]{hyperref}
  \urlstyle{same} % normal text font (alternatives: tt, rm, sf)
}{}
\hypersetup{
  pdftitle={Lösungen zu Gleichungen},
}

%%% Title Data
\title{\phantomsection%
  Lösungen zu Gleichungen%
  \label{id1}%
  \label{losungen-zu-gleichungen}%
  \label{losungen-gleichungen}}
\author{}
\date{}

%%% Body
\begin{document}
\maketitle


\section{Lineare Gleichungen%
  \label{lineare-gleichungen}%
  \label{losungen-lineare-gleichungen}%
}

Die folgenden Lösungen beziehen sich auf die %
\raisebox{1em}{\hypertarget{id3}{}}\hyperlink{id2}{\textbf{\color{red}:ref:`Übungsaufgaben <Aufgaben
Lineare Gleichungen>`}} zum Abschnitt %
\raisebox{1em}{\hypertarget{id5}{}}\hyperlink{id4}{\textbf{\color{red}:ref:`Lineare Gleichungen <Lineare
Gleichungen>`}}.

\DUadmonition[system-message]{
\DUtitle[system-message]{system-message}
\raisebox{1em}{\hypertarget{id2}{}}

{\color{red}ERROR/3} in \texttt{loesungen.rst}, line~13

\hyperlink{id3}{
Unknown interpreted text role \textquotedbl{}ref\textquotedbl{}.
}}

\DUadmonition[system-message]{
\DUtitle[system-message]{system-message}
\raisebox{1em}{\hypertarget{id4}{}}

{\color{red}ERROR/3} in \texttt{loesungen.rst}, line~13

\hyperlink{id5}{
Unknown interpreted text role \textquotedbl{}ref\textquotedbl{}.
}}


%___________________________________________________________________________
\DUtransition

%
\begin{itemize}

\item Zur Lösung der Gleichung empfiehlt es sich, beide Seiten der Gleichung mit dem
Hauptnenner $2 \cdot 3 = 6$ der auftretenden Terme zu multiplizieren.
%
\begin{align*}
\frac{10 \cdot x+3}{3} -5 &= 11 - \frac{3 \cdot x + 4}{2} - \frac{2 \cdot x
+6}{3} \\[4pt]
6 \cdot \left( \frac{10 \cdot x+3}{3} -5 \right) &= 6 \cdot \left( 11 -
\frac{3 \cdot x + 4}{2} - \frac{2 \cdot x +6}{3} \right) \\[4pt]
\end{align*}
Multipliziert man die Klammern aus, so können die auftretenden Brüche durch
Kürzen beseitigt werden. Man erhält dadurch:
%
\begin{align*}
{\color{white}....}2 \cdot (10 \cdot x+3) \; - \; 30  = 66 - 3 \cdot (3
\cdot x + 4) - 2 \cdot (2 \cdot x +6) \\[4pt]
\end{align*}
Die Gleichung kann durch ein Ausmultiplizieren der Klammern weiter vereinfacht
werden:
%
\begin{align*}
20 \cdot x+6 \; - \; 30  = 66 - 9 \cdot x - 12 - 4 \cdot x - 12 \\[4pt]
\end{align*}
Zum Auflösen werden alle $x$-Terme auf eine Seite der Gleichung, alle
anderen Terme auf die andere Seite der Gleichung gebracht. Damit folgt:
%
\begin{align*}
20 \cdot x + 13 \cdot x &= 66 -24 + 24 \\
33 \cdot x &= 66 \\ x &= 2 \\
\end{align*}
Die Lösung der Gleichung lautet somit $x=2$.

%
\raisebox{1em}{\hypertarget{id7}{}}\hyperlink{id6}{\textbf{\color{red}:ref:`Zurück zur Aufgabe <ling01>`}}

\DUadmonition[system-message]{
\DUtitle[system-message]{system-message}
\raisebox{1em}{\hypertarget{id6}{}}

{\color{red}ERROR/3} in \texttt{loesungen.rst}, line~57

\hyperlink{id7}{
Unknown interpreted text role \textquotedbl{}ref\textquotedbl{}.
}}

\end{itemize}


%___________________________________________________________________________
\DUtransition



\section{Quadratische Gleichungen%
  \label{quadratische-gleichungen}%
  \label{losungen-quadratische-gleichungen}%
}

Die folgenden Lösungen beziehen sich auf die %
\raisebox{1em}{\hypertarget{id9}{}}\hyperlink{id8}{\textbf{\color{red}:ref:`Übungsaufgaben <Aufgaben
Quadratische Gleichungen>`}} zum Abschnitt %
\raisebox{1em}{\hypertarget{id11}{}}\hyperlink{id10}{\textbf{\color{red}:ref:`Quadratische Gleichungen
<Quadratische Gleichungen>`}}.

\DUadmonition[system-message]{
\DUtitle[system-message]{system-message}
\raisebox{1em}{\hypertarget{id8}{}}

{\color{red}ERROR/3} in \texttt{loesungen.rst}, line~66

\hyperlink{id9}{
Unknown interpreted text role \textquotedbl{}ref\textquotedbl{}.
}}

\DUadmonition[system-message]{
\DUtitle[system-message]{system-message}
\raisebox{1em}{\hypertarget{id10}{}}

{\color{red}ERROR/3} in \texttt{loesungen.rst}, line~66

\hyperlink{id11}{
Unknown interpreted text role \textquotedbl{}ref\textquotedbl{}.
}}


%___________________________________________________________________________
\DUtransition

%
\begin{itemize}

\item $\text{a) }$ Die Lösungsformel für quadratische Gleichungen
(\textquotedbl{}Mitternachtsformel\textquotedbl{}) liefert für die gegebene Gleichung $x^2 - 6 \cdot
x + 8 = 0$ mit $a=1$, $b=-6$ und $c=8$:
%
\begin{equation*}
x_{1,2} = \frac{-b \pm \sqrt{b^2 - 4 \cdot a \cdot c}}{2 \cdot a} =
\frac{+6 \pm \sqrt{36 - 4 \cdot 8}}{2} = \frac{6 \pm 2}{2}
\end{equation*}
Somit ergeben sich folgende Lösungen:
%
\begin{equation*}
x_1 = \frac{6 - 2}{2} = 2 \qquad x_2 = \frac{6+2}{2} = 4
\end{equation*}
Die Lösungsmenge der Gleichung lautet somit $\mathbb{L} = \{ 2;\, 4 \}$.

\item $\text{b) }$ Zum Lösen der Gleichung $3 \cdot x^2 + 4 \cdot x - 15
= 0$ sind in die \textquotedbl{}Mitternachtsformel\textquotedbl{} die Werte $a=3$, $b=4$ und
$c=-15$ einzusetzen. Man erhält damit:
%
\begin{equation*}
x_{1,2} = \frac{-b \pm \sqrt{b^2 - 4 \cdot a \cdot c}}{2 \cdot a} =
\frac{-4 \pm \sqrt{16 - 4 \cdot 3 \cdot (-15)}}{6} = \frac{-4 \pm
\sqrt{196}}{6}
\end{equation*}
Die Wurzel $\sqrt{196}$ ergibt den Wert $14$. Als Lösungen erhält
man damit:
%
\begin{equation*}
x_1 = \frac{-4 - 14}{6} = -3 \qquad x_2 = \frac{-4+14}{6} = \frac{5}{3}
\end{equation*}
Die Lösungsmenge der Gleichung lautet somit $\mathbb{L} = \left\{ -3;\, \frac{5}{3} \right\}$.

%
\raisebox{1em}{\hypertarget{id13}{}}\hyperlink{id12}{\textbf{\color{red}:ref:`Zurück zur Aufgabe <quag01>`}}

\DUadmonition[system-message]{
\DUtitle[system-message]{system-message}
\raisebox{1em}{\hypertarget{id12}{}}

{\color{red}ERROR/3} in \texttt{loesungen.rst}, line~110

\hyperlink{id13}{
Unknown interpreted text role \textquotedbl{}ref\textquotedbl{}.
}}

\end{itemize}

% x = sy.S('x')

% sy.solve( sy.Eq(3*x**2 + 4*x - 15, 0) )

% [-3, 5/3]

% x = sy.S('x')

% sy.solve( sy.Eq(x**2 - 6*x + 8, 0) )

% [2, 4]


%___________________________________________________________________________
\DUtransition

%
\begin{itemize}

\item Der Satz von Vieta ist insbesondere dann nützlich, wenn eine quadratische
Gleichung der Form $1 \cdot x^2 + b \cdot x + c = 0$ vorliegt und
$b$ sowie $c$ ganze Zahlen sind.

Man prüft dann als erstes, durch welche Produkt zweier Zahlen sich die Zahl
$c$ darstellen lässt. Im Fall $c=20$ ergeben sich folgende
Möglichkeiten:
%
\begin{align*}
20 &= 20 \cdot 1 \\
 &= 10 \cdot 2 \\
 &= 5 \cdot 4 \\
\end{align*}
Ebenfalls möglich sind die Produkte $(-20) \cdot (-1)$, $(-10)
\cdot (-2)$ und $(-5) \cdot (-4)$. Eine dieser drei beziehungsweise
sechs Möglichkeiten gibt die beiden Lösungen der Gleichung an.

Um zu prüfen, welche der obigen Möglichkeiten die Gleichung löst, bildet man
die Summen der einzelnen Wertepaare:
%
\begin{align*}
20 + 1 &= 21 \\
10 + 2 &= 12 \\
5 + 4 &= 9 \\
\end{align*}
Das \textquotedbl{}richtige\textquotedbl{} Wertepaar erkennt man daran, dass die Summe einen Wert ergibt,
der mit dem Wert von $(-b)$ identisch ist. In dieser Aufgabe ist
$b = -9$, also ist $(-b) = 9$. Die Lösung der Gleichung lautet
somit:
%
\begin{equation*}
x_1 = 4 \quad ; \quad x_2 = 5
\end{equation*}
Als Produktform lässt sich die Gleichung damit wie folgt schreiben:
%
\begin{equation*}
x^2 - 9 \cdot x + 20 = 0 \quad \Longleftrightarrow \quad (x-4) \cdot (x-5) = 0
\end{equation*}
%
\raisebox{1em}{\hypertarget{id15}{}}\hyperlink{id14}{\textbf{\color{red}:ref:`Zurück zur Aufgabe <quag02>`}}

\DUadmonition[system-message]{
\DUtitle[system-message]{system-message}
\raisebox{1em}{\hypertarget{id14}{}}

{\color{red}ERROR/3} in \texttt{loesungen.rst}, line~166

\hyperlink{id15}{
Unknown interpreted text role \textquotedbl{}ref\textquotedbl{}.
}}

\end{itemize}


%___________________________________________________________________________
\DUtransition



\section{Algebraische Gleichungen%
  \label{algebraische-gleichungen}%
  \label{losungen-algebraische-gleichungen}%
}

Die folgenden Lösungen beziehen sich auf die %
\raisebox{1em}{\hypertarget{id17}{}}\hyperlink{id16}{\textbf{\color{red}:ref:`Übungsaufgaben <Aufgaben
Algebraische Gleichungen>`}} zum Abschnitt %
\raisebox{1em}{\hypertarget{id19}{}}\hyperlink{id18}{\textbf{\color{red}:ref:`Algebraische Gleichungen höheren
Grades <Algebraische Gleichungen>`}}.

\DUadmonition[system-message]{
\DUtitle[system-message]{system-message}
\raisebox{1em}{\hypertarget{id16}{}}

{\color{red}ERROR/3} in \texttt{loesungen.rst}, line~176

\hyperlink{id17}{
Unknown interpreted text role \textquotedbl{}ref\textquotedbl{}.
}}

\DUadmonition[system-message]{
\DUtitle[system-message]{system-message}
\raisebox{1em}{\hypertarget{id18}{}}

{\color{red}ERROR/3} in \texttt{loesungen.rst}, line~176

\hyperlink{id19}{
Unknown interpreted text role \textquotedbl{}ref\textquotedbl{}.
}}


%___________________________________________________________________________
\DUtransition

%
\begin{itemize}

\item Existiert die Lösung $x_1=3$, so kann der Gleichungsterm in ein Produkt
aus dem Linearfaktor $(x-3)$ und einem Restterm zerlegt werden. Dieser
kann mittels einer Polynom-Division ermittelt werden; es muss also folgende
Rechnung durchgeführt werden:
%
\begin{equation*}
(x^3 - 6 \cdot x^2 - 1 \cdot x + 30) : (x - 3) = \; ?
\end{equation*}
Als erstes prüft man, mit welchem Faktor $x$ zu multiplizieren ist, um
$x^3$ zu erhalten; als Ergebnis kann man $x^2$ auf die rechte
Seite schreiben. Das Produkt aus $x^2 \cdot (x-3)$ muss dann vom
ursprünglichen Term abgezogen werden. Man erhält:
%
\begin{equation*}
\begin{array}{rlll}
(x^3 &-  6 \cdot x^2 &-  1 \cdot x &+  30) : (x - 3) = x^2 \; + \; ?\\
-(x^3 & - 3 \cdot x^2) \\ \cline{1-2} \\[-8pt]
& -3 \cdot x^2 & - 1\cdot x &+ 30 \\
\end{array}
\end{equation*}
Als nächstes ist also zu prüfen, mit welchem Faktor $x$ zu
multiplizieren ist, um $-3 \cdot x^2$ zu erhalten; als Ergebnis kann man
wiederum $-3 \cdot x$ auf die rechte Seite schreiben. Das Produkt aus
$-3 \cdot x \cdot (x-3)$ muss vom verbleibenden Term abgezogen werden.
Man erhält:
%
\begin{equation*}
\begin{array}{rlll}
(x^3 &-  6 \cdot x^2 &-  1 \cdot x &+  30) : (x - 3) = x^2 - 3 \cdot x\\
-(x^3 & - 3 \cdot x^2) \\ \cline{1-2} \\[-8pt]
& -3 \cdot x^2 & - 1 \cdot x &+ 30 \\
\qquad -(& -3 \cdot x^2 & + 9 \cdot x) \\\cline{1-3}\\[-8pt]
&&10 \cdot x & - 30
\end{array}
\end{equation*}
%
\raisebox{1em}{\hypertarget{id21}{}}\hyperlink{id20}{\textbf{\color{red}:ref:`Zurück zur Aufgabe <alg01>`}}

\DUadmonition[system-message]{
\DUtitle[system-message]{system-message}
\raisebox{1em}{\hypertarget{id20}{}}

{\color{red}ERROR/3} in \texttt{loesungen.rst}, line~224

\hyperlink{id21}{
Unknown interpreted text role \textquotedbl{}ref\textquotedbl{}.
}}

\end{itemize}

% sy.expand( (x-3)*(x-5)*(x+2) )

% x**3 - 6*x**2 - x + 30


%___________________________________________________________________________
\DUtransition



\section{Bruch-, Produkt- und Wurzelgleichungen%
  \label{bruch-produkt-und-wurzelgleichungen}%
  \label{losungen-bruch-produkt-und-wurzelgleichungen}%
}

Die folgenden Lösungen beziehen sich auf die %
\raisebox{1em}{\hypertarget{id23}{}}\hyperlink{id22}{\textbf{\color{red}:ref:`Übungsaufgaben <Aufgaben
Bruchgleichungen und Wurzelgleichungen>`}} zum Abschnitt %
\raisebox{1em}{\hypertarget{id25}{}}\hyperlink{id24}{\textbf{\color{red}:ref:`Bruch-, Produkt-
und Wurzelgleichungen <Bruchgleichungen und Wurzelgleichungen>`}}.

\DUadmonition[system-message]{
\DUtitle[system-message]{system-message}
\raisebox{1em}{\hypertarget{id22}{}}

{\color{red}ERROR/3} in \texttt{loesungen.rst}, line~240

\hyperlink{id23}{
Unknown interpreted text role \textquotedbl{}ref\textquotedbl{}.
}}

\DUadmonition[system-message]{
\DUtitle[system-message]{system-message}
\raisebox{1em}{\hypertarget{id24}{}}

{\color{red}ERROR/3} in \texttt{loesungen.rst}, line~240

\hyperlink{id25}{
Unknown interpreted text role \textquotedbl{}ref\textquotedbl{}.
}}

\DUrubric{Bruch- und Produktgleichungen}


%___________________________________________________________________________
\DUtransition

%
\begin{itemize}

\item Bei der Gleichung handelt es sich um eine Produkt-Gleichung; für den
Definitionsbereich gilt $\mathbb{D} = \mathbb{R}$, es dürfen also alle
reellen Zahlen für $x$ eingesetzt werden.

Um die Gleichung zu lösen, ist es hilfreich, alle die Variable $x$
beinhaltende Terme auf die linke Seite zu sortieren. Dadurch erhält man:
%
\begin{align*}
3 \cdot x \cdot (x - 5) &= 6 \cdot (x - 5) \\
3 \cdot x \cdot (x - 5) - 6 \cdot (x - 5)&= 0
\end{align*}
Auf der linken Seite kann nun der Term $(x-5)$ ausgeklammert werden.
Daraus ergibt sich:
%
\begin{equation*}
(3 \cdot x - 6) \cdot (x - 5) &= 0
\end{equation*}
Ein Produkt hat genau dann den Wert Null, wenn einer der Faktoren Null ist.
Die Gleichung ist somit in den folgenden beiden Fällen erfüllt:
%
\begin{align*}
3 \cdot x - 6 &= 0 \quad \Longleftrightarrow \quad x = 2 \\
x - 5 &= 0 \quad \Longleftrightarrow \quad x = 5
\end{align*}
Die Lösungsmenge der Gleichung ist somit $\mathbb{L} = \{2;\,5\}$.

\textbf{Hinweis:} Würde man im ersten Schritt durch $(x-5)$ dividieren, so
bliebe nur noch die Lösung $x=2$ übrig. Bei einer Division einer
Gleichung durch einen Term muss also stets darauf geachtet werden, dass dieser
Term ungleich Null ist; gegebenenfalls muss eine Fallunterscheidung
vorgenommen und dieser Fall -{}- im obigen Beispiel $x=5$ -{}- separat
untersucht werden.

%
\raisebox{1em}{\hypertarget{id27}{}}\hyperlink{id26}{\textbf{\color{red}:ref:`Zurück zur Aufgabe <prog01>`}}

\DUadmonition[system-message]{
\DUtitle[system-message]{system-message}
\raisebox{1em}{\hypertarget{id26}{}}

{\color{red}ERROR/3} in \texttt{loesungen.rst}, line~289

\hyperlink{id27}{
Unknown interpreted text role \textquotedbl{}ref\textquotedbl{}.
}}

\end{itemize}


%___________________________________________________________________________
\DUtransition

%
\begin{itemize}

\item Bei Bruchgleichungen muss ausgeschlossen sein, dass die Nenner der
auftretenden Terme gleich Null werden; es muss also gelten:
%
\begin{align*}
2 \cdot x + 10 \ne 0 \quad &\Longleftrightarrow \quad x \ne -5 \text{ und }\\
4 - 2 \cdot x \ne 0 \quad &\Longleftrightarrow \quad x \ne 2
\end{align*}
Um die Gleichung zu lösen, ist es empfehlenswert, beide Seiten der Gleichung
mit dem Hauptnenner $(2 \cdot x + 10) \cdot (4 - 2 \cdot x)$ zu
multiplizieren. Nach dem Kürzen entfallen dadurch die Nenner:
%
\begin{align*}
\frac{3 \cdot x + 13}{2 \cdot x + 10} &= \frac{4 - 3 \cdot x}{4 - 2\cdot
x} \\[8pt]
\frac{\cancel{(2 \cdot x + 10)} \cdot (4 - 2 \cdot x) \cdot (3 \cdot x +
13)}{\cancel{2 \cdot x + 10}} &= \frac{(4 - 3 \cdot x) \cdot (2 \cdot x +
10) \cdot \cancel{(4 - 2 \cdot x)}}{\cancel{4 - 2\cdot x}} \\[8pt]
\Rightarrow (4 - 2 \cdot x) \cdot (3 \cdot x + 13) &= (4 - 3 \cdot x)
\cdot (2 \cdot x + 10)
\end{align*}
Um diese Gleichung weiter zu vereinfachen, müssen die Terme auf beiden Seiten
ausmultipliziert werden. Man erhält:
%
\begin{align*}
12 \cdot x + 52 - 6 \cdot x^2 - 26 \cdot x &= 8 \cdot x + 40 - 6 \cdot x^2
- 30 \cdot x \\
- 6 \cdot x^2 -14 \cdot x + 52 &= -6 \cdot x^2 -22 \cdot x + 40 \\
\end{align*}
Sortiert man alle $x$-Terme auf die linke und alle übrigen Terme auf die
rechte Seite, so entfällt der quadratische Term. Übrig bleibt eine lineare
Gleichung mit folgender Lösung:
%
\begin{align*}
8 \cdot x \; &= -12 \\
x \; &= -\frac{3}{2} \\
\end{align*}
Die Lösungsmenge der Gleichung ist somit $\mathbb{L} = \{ -\frac{3}{2} \}$.

%
\raisebox{1em}{\hypertarget{id29}{}}\hyperlink{id28}{\textbf{\color{red}:ref:`Zurück zur Aufgabe <bru01>`}}

\DUadmonition[system-message]{
\DUtitle[system-message]{system-message}
\raisebox{1em}{\hypertarget{id28}{}}

{\color{red}ERROR/3} in \texttt{loesungen.rst}, line~337

\hyperlink{id29}{
Unknown interpreted text role \textquotedbl{}ref\textquotedbl{}.
}}

\end{itemize}


%___________________________________________________________________________
\DUtransition


\DUrubric{Wurzelgleichungen}


%___________________________________________________________________________
\DUtransition

%
\begin{itemize}

\item Betrachtet man (ohne jegliche algebraische Umformung) den Definitionsbereich
der Gleichung, so stellt man fest, dass dieser leer ist. Es gibt nämlich
keinen Wert für die Variable $x$, so dass die beiden Bedingungen
$x-5 \ge 0$ und $2-x \ge 0$ gleichzeitig erfüllt sind. Da dies
nicht möglich ist, kann die Gleichung folglich für keine reelle Zahl $x
\in \mathbb{R}$ erfüllt werden.

%
\raisebox{1em}{\hypertarget{id31}{}}\hyperlink{id30}{\textbf{\color{red}:ref:`Zurück zur Aufgabe <wurz01>`}}

\DUadmonition[system-message]{
\DUtitle[system-message]{system-message}
\raisebox{1em}{\hypertarget{id30}{}}

{\color{red}ERROR/3} in \texttt{loesungen.rst}, line~356

\hyperlink{id31}{
Unknown interpreted text role \textquotedbl{}ref\textquotedbl{}.
}}

\end{itemize}


%___________________________________________________________________________
\DUtransition

%
\begin{itemize}

\item Die Definitionsmenge ergibt sich, da reellwertige Wurzeln nicht negativ sein
dürfen, aus folgenden Ungleichungen:
%
\begin{align*}
\sqrt{x + 1} \ge 0 \quad &\Longleftrightarrow \quad x \ge -1 \\
x - 5 \ge 0 \quad &\Longleftrightarrow \quad x \ge 5 \\
\end{align*}
Da beide Bedingungen zugleich gelten müssen und die zweite Bedingung $x
\ge 5$ die erste Bedingung $x \ge -1$ hinreichend mit einschließt, gilt
für den Definitionsbereich der Gleichung $\mathbb{D} = [5; \infty[$.

Um die Gleichung zu lösen, können die Terme auf beiden Seiten in einem ersten
Rechenschritt quadriert werden. Man erhält hierbei:
%
\begin{equation*}
x + 1 = (x - 5)^2
\end{equation*}
Diese Gleichung entspricht nun einer quadratischen Gleichung. Um sie zu lösen,
werden alle Terme auf die linke Seite sortiert und anschließend Klammer der
quadratische Term $(x-5)^2$ ausgewertet:
%
\begin{align*}
(x-5)^2 \quad - x - 1 &= 0 \\
(x^2 - 10 \cdot x + 25) - x - 1 &= 0 \\
\end{align*}
Da in der resultierenden Gleichung alle Operatoren die gleiche Priorität haben
und vor der Klammer kein Minuszeichen steht, können die Klammern weggelassen
werden. Die $x$-Terme sowie die Zahlenwerte können noch folgendermaßen
zusammengefasst werden:
%
\begin{align*}
x^2 - 11 \cdot x + 24 \quad  &= 0 \\
\end{align*}
Diese Gleichung kann beispielsweise mit der Lösungsformel für quadratische
Gleichungen gelöst werden. Mit $a=1$, $b=-11$ und $c=24$
erhält man:
%
\begin{align*}
x_1 = \frac{-b - \sqrt{b^2 - 4 \cdot a \cdot c}}{2 \cdot a} = \frac{11 -
\sqrt{121 - 4 \cdot 24}}{2 \cdot 1} = \frac{11 - \sqrt{25}}{2} = 3\\
x_1 = \frac{-b + \sqrt{b^2 - 4 \cdot a \cdot c}}{2 \cdot a} = \frac{11 +
\sqrt{121 - 4 \cdot 24}}{2 \cdot 1} = \frac{11 + \sqrt{25}}{2} = 8\\
\end{align*}
Man könnte nun annehmen, dass die Lösungsmenge gleich $\mathbb{L} = \{
3;\,5 \}$ ist -{}- doch das ist falsch! Die Definitionsmenge $\mathbb{D} =
[5;\,\infty[$ der ursprünglichen Gleichung schließt die Lösung $x_1 = 3$
der späteren quadratischen Gleichung aus. Der Grund für das Hinzukommen der
\textquotedbl{}Scheinlösung\textquotedbl{} liegt im ersten Rechenschritt, nämlich dem Quadrieren beider
Seiten der Gleichung. Da diese Umformung keine Äquivalenz-Umformung ist,
können -{}- wie in diesem Beispiel -{}- weitere Lösungen hinzukommen.

Neben einem Blick auf den Definitionsbereich schließt auch ein Einsetzen der
erhaltenen Lösungen in die ursprüngliche Gleichung die Scheinlösung
$x_1=3$ aus. Die Lösungsmenge lautet also $\mathbb{L} = \{ 8 \}$.

%
\raisebox{1em}{\hypertarget{id33}{}}\hyperlink{id32}{\textbf{\color{red}:ref:`Zurück zur Aufgabe <wurz02>`}}

\DUadmonition[system-message]{
\DUtitle[system-message]{system-message}
\raisebox{1em}{\hypertarget{id32}{}}

{\color{red}ERROR/3} in \texttt{loesungen.rst}, line~422

\hyperlink{id33}{
Unknown interpreted text role \textquotedbl{}ref\textquotedbl{}.
}}

\end{itemize}


%___________________________________________________________________________
\DUtransition



\section{Exponential- und Logarithmusgleichungen%
  \label{exponential-und-logarithmusgleichungen}%
  \label{losungen-exponential-und-logarithmusgleichungen}%
}

Die folgenden Lösungen beziehen sich auf die %
\raisebox{1em}{\hypertarget{id35}{}}\hyperlink{id34}{\textbf{\color{red}:ref:`Übungsaufgaben <Aufgaben
Exponential- und Logarithmusgleichungen>`}} zum Abschnitt %
\raisebox{1em}{\hypertarget{id37}{}}\hyperlink{id36}{\textbf{\color{red}:ref:`Exponential- und
Logarithmusgleichungen <Exponential- und Logarithmusgleichungen>`}}.

\DUadmonition[system-message]{
\DUtitle[system-message]{system-message}
\raisebox{1em}{\hypertarget{id34}{}}

{\color{red}ERROR/3} in \texttt{loesungen.rst}, line~432

\hyperlink{id35}{
Unknown interpreted text role \textquotedbl{}ref\textquotedbl{}.
}}

\DUadmonition[system-message]{
\DUtitle[system-message]{system-message}
\raisebox{1em}{\hypertarget{id36}{}}

{\color{red}ERROR/3} in \texttt{loesungen.rst}, line~432

\hyperlink{id37}{
Unknown interpreted text role \textquotedbl{}ref\textquotedbl{}.
}}


%___________________________________________________________________________
\DUtransition

%
\begin{itemize}

\item Die Definitionsmenge der Gleichung ist $\mathbb{D} = \{ x \,|\, x > 0,\;
x \ne 1 \}$.

Gemäß der Definition eines Logarithmus kann die Gleichung auch wie folgt
geschrieben werden:
%
\begin{equation*}
\log_{x}{(125)} = 3 \quad \Longleftrightarrow \quad x^{3} = 125
\end{equation*}
Zieht man bei der Gleichung auf der rechten Seite die dritte Wurzel, so erhält
man:
%
\begin{equation*}
x = \sqrt[3]{125} = \pm 5
\end{equation*}
Unter Berücksichtigung der Definitionsmenge lautet die Lösung somit
$\mathbb{L} = \{ 5 \}$.

%
\raisebox{1em}{\hypertarget{id39}{}}\hyperlink{id38}{\textbf{\color{red}:ref:`Zurück zur Aufgabe <gel01>`}}

\DUadmonition[system-message]{
\DUtitle[system-message]{system-message}
\raisebox{1em}{\hypertarget{id38}{}}

{\color{red}ERROR/3} in \texttt{loesungen.rst}, line~460

\hyperlink{id39}{
Unknown interpreted text role \textquotedbl{}ref\textquotedbl{}.
}}

\end{itemize}


%___________________________________________________________________________
\DUtransition


% .

\DUadmonition[system-message]{
\DUtitle[system-message]{system-message}


{\color{red}ERROR/3} in \texttt{loesungen.rst}, line~466

Unknown directive type \textquotedbl{}only\textquotedbl{}.
%
\begin{quote}{\ttfamily \raggedright \noindent
..~only::~html\\
~\\
~~~~:ref:`Zurück~zum~Skript~<Exponential-~und~Logarithmusgleichungen>`\\
~\\

}
\end{quote}
backrefs: }

\end{document}
